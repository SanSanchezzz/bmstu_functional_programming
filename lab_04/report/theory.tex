\chapter{Теоретические сведения}

\textbf{Базис} в Lisp образуют:
\begin{itemize}
    \item атомы;
    \item структуры;
    \item базовые функции;
    \item базовые функционалы.
\end{itemize}

Функция в Лиспе есть однозначное отображение множества исходных данных на множество её значений.
У функции может быть произвольно много аргументов, от нуля до любого конечного числа, но обязательно должно быть хотя бы одно значение.

\textbf{Классификация функций:}
\begin{itemize}
    \item \textbf{Базовые функции} – принимают фиксированное количество аргументов
    \item \textbf{Формы} – принимают не фиксированное количество аргументов или обрабатывают аргументы по разному
    \item \textbf{Функционалы (высших порядков)} – используют другие функции в качестве аргументов или вырабатывают в качестве результатов.
\end{itemize}

\textbf{CAR} и \textbf{CDR} являются базовыми функциями доступа к данным.
CAR принимает точечную пару или пустой список в качестве аргумента и возвращает первый элемент или nil, соответственно.
CDR принимает точечную пару или пустой список и возвращает список состоящий из всех элементов, кроме первого. Если в списке меньше двух элементов, то возвращается Nil.

\textbf{LIST} и \textbf{CONS} являются функциями создания списков (cons – базовая, list – нет).
Функция cons создает списочную ячейку и устанавливает два указателя на аргументы. Функция list принимает переменное число аргументов и возвращает список, элементы которого – переданные в функцию аргументы.
