\chapter{Теоретические сведения}

\textbf{Базис} в Lisp образуют:
\begin{itemize}
    \item атомы;
    \item структуры;
    \item базовые функции;
    \item базовые функционалы.
\end{itemize}

Вся информация (данные и программы) в Lisp представляется в виде S-выражений. \textbf{S-выражения} - это атом или точечная пара.
\textbf{Атомами} являются:
\begin{itemize}
    \item \textbf{символы} -  (идентификаторы) –  синтаксически – набор литер (букв и цифр), начинающихся с буквы;
    \item \textbf{специальные символы – {Т, Nil}} - (используются для обозначения логических констант);
    \item \textbf{самоопределяемые атомы} - натуральные числа, дробные числа (например 2/3), вещественные числа, строки – последовательность символов, заключенных в двойные апострофы (например “abc”);
\end{itemize}

Более сложные данные — списки и точечные пары (структуры) строятся из унифицированных структур – блоков памяти – бинарных узлов.

\textbf{Точечной парой} является конструкция вида (A . B), где под A и B подразумевается либо атом, либо точечная пара.

\textbf{Список} быть пустым или не пустым. \textbf{Пустой список} - это () или, как было упомянуто выше, Nil. \textbf{Не пустой список} - это точечная пара, состоящая из головы и хвоста. \textbf{Голова списка} - это S-выражение, а \textbf{хвост} - список.

