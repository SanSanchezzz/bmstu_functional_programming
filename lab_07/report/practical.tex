\chapter{Практическая часть}

\section{Задание №1}

Чем принципиально отличаются функции cons, lisp, append?

\begin{lstlisting}
(setf lst1 '(a b))
(setf lst2 '(c d))

(cons lst1 lst2)   ;;; ((A B) C D)
(list lst1 lst2)   ;;; ((A B) (C D))
(append lst1 lst2) ;;; (A B C D)
\end{lstlisting}

Функция cons создает списковую ячейку и кладет в голову первый аргумент, а в хвост --
второй. Функция list создает списковые ячейки для каждого аргумента и соединяет
их в один список. А функция append объединяет два списка в один, состоящий из
элементов двух списков.

\section{Задание №2}

Каковы результаты вычисления следующих выражений?

\begin{lstlisting}
(reverse ())         ;;; Nil
(last ())            ;;; Nil
(reverse '(a))       ;;; (A)
(last '(a))          ;;; (A)
(reverse '((a b c))) ;;; ((A B C))
(last '((a b c)))    ;;; ((A B C))
\end{lstlisting}

\section{Задание №3}

Написать два варианта функции, которая возвращает последний элемент своего
списка-аргумента.

Рекурсивно -- пока второй элемент не Nil идем дальше, иначе возвращаем голову:

\begin{lstlisting}
(defun last_elem (lst)
  (if (NULL (cadr lst)) (car lst)
    (last_elem (cdr lst))
  )
)
\end{lstlisting}

С использованием функционала -- с использованием функционала reduce, возвращая последний полученный результат, а возвращаем каждый раз второй аргумент:

\begin{lstlisting}
(defun last_elem (lst)
  (reduce #'(lambda (a x) x) lst)
)
\end{lstlisting}

\section{Задание №4}
Написать два варианта функции, которая возвращает свой список-аргумент без
последнего элемента.

С использованием функционала -- пользуясь тем, что mapcar будет работать до хвоста в самом коротком списке (cdr lst всегда на один короче lst), возвращаем значение из lst, таким образом получится список без последнего элемента:

\begin{lstlisting}
(defun centr (lst)
  (mapcar (lambda (x y) y) (cdr lst) lst)
)
\end{lstlisting}

Рекурсивно -- пока хвост не Nil, соединяем с помощью cons голову и возвращаем значение функции от хвоста:

\begin{lstlisting}
(defun centr (lst)
  (if (NULL (cadr lst)) ()
    (cons (car lst) (centr (cdr lst)))
  )
)
\end{lstlisting}

\section{Задание №5}
Написать простой вариант игры в кости, в котором бросаются две правильные
кости. Если сумма выпавших очков равна 7 или 11 -- выигрыш, если выпало
(1, 1) или (6, 6) -- игрок получает право снова бросить кости, во всех остальных
случаях ход переходит ко второму игроку, но запоминается сумма выпавших
очков. Если второй игрок не выигрывает абсолютно, то выигрывает тот игрок,
у которого больше очков. Результат игры и значения выпавших костей выводить
на экран с помощью функции print.

\begin{lstlisting}
(defun print (x y s)
  (print (list x '+ y '= s))
)

(defun analyze (x y)
  (cond
    ((and (= 1 x) (= 1 y)) (print x y (+ x y))
        (analyze (+ 1 (random 6)) (+ 1 (random 6))))
    ((and (= 6 x) (= 6 y)) (print x y (+ x y))
        (analyze (+ 1 (random 6)) (+ 1 (random 6))))
    (T (printS x y (+ x y)) (+ x y))
  )
)

(defun main ()
  (let ((s1 (analyze (+ 1 (random 6)) (+ 1 (random 6))))
        (s2 (analyze (+ 1 (random 6)) (+ 1 (random 6)))))
    (cond
      ((or (= s1 7) (= s1 11) (> s1 s2)) 'first-win)
      ((or (= s1 7) (= s2 11) (< s1 s2)) 'second-wins)
      (T 'draw)
    )
  )
)
\end{lstlisting}
