\chapter{Практическая часть}

\section{Задание \No{}1}

Написать функцию, которая принимает целое число и возвращает первое
четное число, не меньшее аргумента.

\begin{lstlisting}
(defun first_even_after (num)
    (if (oddp num)
        (+ num 1)
        num
    )
)
\end{lstlisting}

\section{Задание \No{}2}


Написать функцию, которая принимает число и возвращает число
того же знака, но с модулем на 1 больше модуля аргумента.

\begin{lstlisting}
(defun abs_plus_one (num)
    (if (> num 0)
        (+ num 1)
        (- num 1)
    )
)
\end{lstlisting}

\section{Задание \No{}3}

Написать функцию, которая принимает два числа и возвращает
список из этих чисел, расположенный по возрастанию.

\begin{lstlisting}
(defun sort_two_numbers (n1 n2)
    (if (< n1 n2)
        (list n1 n2)
        (list n2 n1)
    )
)
\end{lstlisting}

\section{Задание \No{}4}

Написать функцию, которая принимает три числа и возвращает Т только
тогда, когда первое число расположенно между вторым и третьим.

\begin{lstlisting}
(defun between_two_numbers (a b c)
    (and (<= b a) (>= c a))
)
\end{lstlisting}

\section{Задание \No{}5}

Каков результат вычисления следующих выражений?

\begin{lstlisting}
(and 'fee 'fie 'foe)
;;; Результат: FOE

(or nil 'fie 'foe)
;;; Результат: FIE

(and (equal 'abc 'abc) 'yes)
;;; Результат: YES

(or 'fee 'fie 'foe)
;;; Результат: FEE

(and nil 'fie 'foe)
;;; Результат: NIL

(or (equal 'abc 'abc) 'yes)
;;; Результат: T
\end{lstlisting}

\section{Задание \No{}6}

Написать предикат, который принимает два числа-аргумента и возвращает
Т, если первое число не меньше второго.

\begin{lstlisting}
(defun more_or_equal (num1 num2)
    (>= num1 num2)
)
\end{lstlisting}

\section{Задание \No{}7}

Какой из следующих двух вариантов предиката ошибочен и почему?

\begin{lstlisting}
(defun pred1 (x)
    (and (numberp x) (plusp x))
)

(defun pred2 (x)
    (and (plusp x) (numberp x))
)

(pred1 'a) ;;; Nil
(pred2 'a) ;;; Error
\end{lstlisting}

В варианте (pred2 'a) возникает ошибка при проверке аргумента на положительность. Необходимо проверять, является ли входной параметр числом.

\section{Задание \No{}8}

Решить задачу 4, используя для ее решения конструкции
IF, COND, AND/OR.

\begin{lstlisting}
(defun between_two_numbers (num num1 num2)
    (if (or
            (and (< num num1) (> num num2))
            (and (> num num1) (< num num2))
        )
        t
        nil
    )
)
\end{lstlisting}

