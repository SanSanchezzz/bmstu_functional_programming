\chapter{Практическая часть}

\section{Задание \No{}1}
Пусть list-of-list список, состоящий из списков. Написать функцию, которая вычисляет сумму длин всех элементов list-of-list.
Например для аргумента ((1 2) (3 4)) -> 4.

\begin{lstlisting}
(defun len_lists (lst)
    (cond
        ((null lst) 0)
        (t (+ (length (car lst))
              (len_lists (cdr lst))))
    )
)
\end{lstlisting}

\section{Задание \No{}2}
Написать рекурсивную версию (с именем reg-add) вычисления суммы чисел
заданного списка.
Например: (reg-add (2 4 6)) -> 12

\begin{lstlisting}
(defun reg-add (lst)
    (cond
        ((null lst) 0)
        (t (+ (car lst) (reg-add (cdr lst))))
    )
)
\end{lstlisting}

\section{Задание \No{}3}
Написать рекурсивную версию с именем recnth функции nth.

\begin{lstlisting}
(defun recnth (n lst)
    (cond
        ((or (= 0 n)
             (null lst)) (car lst))
        (t (recnth (- n 1) (cdr lst)))
    )
)
\end{lstlisting}

\section{Задание \No{}4}
Написать рекурсивную функцию alloddr, которая возвращает t когда все
элементы списка нечетные.

\begin{lstlisting}
(defun alloddr (lst)
    (cond
        ((null lst) t)
        ((eql (mod (car lst) 2) 0) nil)
        (t (alloddr (cdr lst)))
    )
)
\end{lstlisting}

\section{Задание \No{}5}

Написать рекурсивную функцию, относящуюся к хвостовой рекурсии с одним тестом завершения, которая возвращает последний элемент списка - аргументы.

\begin{lstlisting}
(defun latest (lst)
    (cond
        ((null lst) nil)
        ((eql (length lst) 1) (car lst))
        (t (latest (cdr lst)))
    )
)
\end{lstlisting}

\section{Задание \No{}6}

Написать рекурсивную функцию, относящуюся к дополняемой рекурсии с
одним тестом завершения, которая вычисляет сумму всех чисел от 0 до n-ого аргумента функции.
Вариант:
\begin{enumerate}
    \item  от п-аргумента функции до последнего >= 0,
    \item  от п-аргумента функции до т-аргумента с шагом d.
\end{enumerate}

Дополняющая рекурсия, считающая сумму всех чисел от 0 до n-го аргумента.
\begin{lstlisting}
(defun sum_first(lst n)
  (cond
    ((= n (length lst)) (car lst))
    (t (+ (car lst) (sum_first (cdr lst) n)))
  )
)

(defun sum_first_n(lst n)
  (sum_first lst (+ 1 (- (length lst) n)))
)
\end{lstlisting}

Дополняющая рекурсия, считающая сумму всех чисел от n-го аргумента до конца.
\begin{lstlisting}
(defun sum_last(lst n)
  (cond
    ((= 1 (length lst)) (car lst))
    ((>= n (length lst)) (+ (car lst)
    	(sum_last (cdr lst) n))
    )
    (t (sum_last (cdr lst) n))
  )
)

(defun sum_last_n(lst n)
  (sum_last lst (+ 1 (- (length lst) n)))
)
\end{lstlisting}

Дополняющая рекурсия, считающая сумму всех чисел от n-го аргумента до m-го с шагом d.
\begin{lstlisting}
(defun sum_range(lst n m d)
  (let ((len (length lst)))
    (cond
      ((= m len)
        (if (= 0 (rem (- n m) 3))
          (car lst)
          0
        )
      )
      ((and (>= n len) (= 0 (rem (- n len) 3)))
        (+ (car lst) (sum_range (cdr lst) n m d))
      )
      (t
        (sum_range (cdr lst) n m d)
      )
    )
  )
)

(defun sum_range_nmd(lst n m d)
  (let ((len (length lst)))
    (sum_range lst (+ 1 (- len n)) (+ 1 (- len m)) d)
  )
)
\end{lstlisting}

\section{Задание \No{}7}

Написать рекурсивную функцию, которая возвращает последнее нечетное
число из числового списка, возможно создавая некоторые вспомогательные функции.

\begin{lstlisting}
(defun last_odd_helper (lst lodd)
  (cond ((null lst) lodd)
        (t (last_odd_helper (cdr lst)
                  (cond ((oddp (car lst)) (car lst))
                        (t lodd))))
  )
)

(defun last_odd (lst)
    (last_odd_helper lst nil)
)
\end{lstlisting}

\section{Задание \No{}8}

Используя cons-дополняемую рекурсию с одним тестом завершения,
написать функцию которая получает как аргумент список чисел, а возвращает список квадратов этих чисел в том же порядке.

\begin{lstlisting}
(defun square_helper (lst)
    (cons (* (car lst) (car lst))
            (if (> (length (cdr lst)) 0)
                (square_helper (cdr lst))
                nil
            )
    )
)

(defun square (lst)
    (if (null lst)
        nil
        (square_helper lst)
    )
)
\end{lstlisting}

\section{Задание \No{}9}

Написать функцию с именем select-odd, которая из заданного
списка выбирает все нечетные числа.
\begin{enumerate}
    \item Вариант 1: select-even,
    \item Вариант 2: вычисляет сумму всех нечетных чисел(sum-all-odd) или сумму всех четных чисел (sum-all-even) из заданного списка. )
\end{enumerate}

\begin{lstlisting}
(defun filter (predicate lst)
  (cond ((null lst) nil)
        (t (cond ((funcall predicate (car lst))
                  (cons (car lst) (filter predicate (cdr lst))))
                 (t (filter predicate (cdr lst)))
           )
        )
  )
)

(defun select_odd (nums)
  (filter #'oddp nums)
)

(defun select_even (nums)
  (filter #'evenp nums)
)

(defun sum_all_odd (nums)
  (reduce #'+ (select_odd nums))
)

(defun sum_all_even (nums)
  (reduce #'+ (select_even nums))
)
\end{lstlisting}
