\chapter{Практическая часть}

\section{Задание \No{}1}

Написать предикат set-equal, который возвращает t, если два его множество-
аргумента содержат одни и те же элементы, порядок которых не имеет значения.

\begin{lstlisting}
(defun set-equal (lst1 lst2)
    (and (subsetp lst1 lst2) (subsetp lst2 lst1))
\end{lstlisting}

\section{Задание \No{}2}
Напишите необходимые функции, которые обрабатывают таблицу из точечных пар:
(страна. столица), и возвращают по стране - столицу, а по столице - страну.

Создаём точеченые пары:

\begin{lstlisting}
(setq cities '(Berlin Paris Moscow London))
(setq countries '(Germany France Russia England))

(defun set_points (a b)
  (cons a b)
)

(setq points (mapcar #'set_points countries cities))
\end{lstlisting}

Рекурсивный поиск:

\begin{lstlisting}
(defun found_country (city lst)
  (cond ((null lst) nil)
        ((eql city (cdr (car lst))) (caar lst))
        (t (found_country city (cdr lst)))
  )
)

(defun found_city (country lst)
  (cond ((null lst) nil)
        ((eql country (car (car lst))) (cdr (car lst)))
        (t (found_city country (cdr lst)))
  )
)
\end{lstlisting}

Поиск с использование функционалов:

\begin{lstlisting}
(defun found_country_func (city lst)
  (defun found (lst1 lst2)
    (if (consp lst1)
      (or (if (eql city (cdr lst1)) (car lst1) Nil)
          (if (eql city (cdr lst2)) (car lst2) Nil) )
      (or lst1 (if (eql city (cdr lst2)) (car lst2) Nil) )
    )
  )
  (reduce #'found newPoints)
)

(defun found_city_func (country lst)
  (defun found (lst1 lst2)
    (if (consp lst1)
      (or (if (eql country (car lst1)) (cdr lst1) Nil)
          (if (eql country (car lst2)) (cdr lst2) Nil) )
      (or lst1 (if (eql country (car lst2))(cdr lst2) Nil))
    )
  )
  (reduce #'found newPoints)
)
\end{lstlisting}


\section{Задание \No{}3}

Напишите функцию, которая умножает на заданное число-аргумент все числа
из заданного списка-аргумента, когда
\begin{enumerate}
\item все элементы списка --- числа,
\item элементы списка -- любые объекты.
\end{enumerate}

С использованием рекурсии:

\begin{lstlisting}
(defun mul_num_rec (lst k)
  (if lst
    (cons (* (car lst) k) (mul_num_rec (cdr lst) k))
  )
)

(defun mul_all_rec (lst k)
  (if lst
    (cons
      (if (numberp (car lst)) (* (car lst) k) (car lst))
      (mul_all_rec (cdr lst) k)
    )
  )
)
\end{lstlisting}

С использованием функционалов:

\begin{lstlisting}
(defun mul_num (lst k)
  (mapcar #'(lambda (x) (* x k)) lst)
)

(defun mul_all (lst k)
  (mapcar #'(lambda (x) (if (numberp x) (* x k) x)) lst)
)
\end{lstlisting}

\section{Задание \No{}4}

Напишите функцию, которая уменьшает на 10 все числа из списка
аргумента этой функции.

С использованием рекурсии:
\begin{lstlisting}
(defun minus_ten_rec (lst)
    (cond
        ((null lst) Nil)
        ('T (cons (- (car lst) 10) (minus_ten_car (cdr lst))))
    )
)
\end{lstlisting}

С использованием функционалов:
\begin{lstlisting}
(defun minus_ten_car (lst)
    (mapcar #'(lambda (el) (- el 10)) lst)
)
\end{lstlisting}

\section{Задание \No{}5}

Написать функцию, которая возвращает первый аргумент списка -аргумента.
который сам является непустым списком.

С использованием рекурсии:
\begin{lstlisting}
(defun first_list_rec (lst)
    (cond
        ((null lst) Nil)
        ((not (atom (car lst))) (car lst))
        ('T (first_list_rec (cdr lst)))
    )
)
\end{lstlisting}

С использованием функционалов:
\begin{lstlisting}
(defun first_list_map (lst)
    (reduce #'
        (lambda (el1 el2)
            (or
                (and (not (atom el1)) el1)
                (and (not (atom el2)) el2)
            )
        ) lst
    )
)
\end{lstlisting}


\section{Задание \No{}6}

Написать функцию, которая выбирает из заданного списка только те числа,
которые больше 1 и меньше 10.
(Вариант: между двумя заданными границами.)

С использованием рекурсии:

\begin{lstlisting}
(defun found_between_rec (a b lst)
  (if lst
    (append
      (if (and (< a (car lst)) (< (car lst) b))
      	(list (car lst)) Nil
      )
      (found_between_rec a b (cdr lst))
    )
  )
)
\end{lstlisting}

С использованием функционалов:

\begin{lstlisting}
(defun found_between (a b lst)
  (defun found (lst1 lst2)
    (if (numberp lst1)
      (append(if(and(< a lst1)(< lst1 b))(list lst1) Nil)
              (if(and(< a lst2)(< lst2 b))(list lst2) Nil))
      (append lst1
              (if(and(< a lst2)(< lst2 b))(list lst2) Nil))
    )
  )
  (reduce #'found lst)
)
\end{lstlisting}

\section{Задание \No{}7}

Написать функцию, вычисляющую декартово произведение двух своих списков-аргументов. ( Напомним, что А х В это множество всевозможных пар (a b), где а принадлежит А, принадлежит В.)

С использованием рекурсии:

\begin{lstlisting}
(defun decart (el lst)
    (cond
        ((null lst) Nil)
        ('T (cons (cons el (car lst)) (decart_one_element el (cdr lst))))
    )
)

(defun decart_rec (lst1 lst2)
    (cond
        ((null lst1) Nil)
        ('T (append (decart (car lst1) lst2) (decart_rec (cdr lst1) lst2)))
    )
)
\end{lstlisting}

С использованием функционалов:

\begin{lstlisting}
(defun decart_map (lst1 lst2)
    (mapcan #'
        (lambda (x)
            (mapcar #'(lambda (y) (cons x y)) lst2)
        ) lst1
    )
)
\end{lstlisting}

\section{Задание \No{}8}

Почему так реализовано reduce, в чем причина?

\begin{lstlisting}
(reduce #'+ ()) -> 0
\end{lstlisting}

Результатом функции будет значение по умолчанию (то есть 0), так как список, к которому применяется функция является пустым.
