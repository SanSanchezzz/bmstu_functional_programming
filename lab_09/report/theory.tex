\chapter{Теоретические сведения}

\section{Способы организации повторных вычислений в Lisp}

\begin{itemize}
    \item использование функционалов
    \item использование рекурсии
\end{itemize}

\section{Различные способы использования функционалов}

{\ttfamily mapcar} -- функция func применяется ко всем элементам списка, начиная
с первого.

{\ttfamily maplist} -- функция func применяется ко всем элементам списка, начиная
с последнего.

{\ttfamily mapcan, mapcon} -- аналогичны mapcar и maplist, используется память
исходных данных, не работают с копиями.

{\ttfamily reduce} -- функция func применяется каскадным образом
(сначала для первого и второго элемента, потом для результата и следующего и т.д.).

\section{Что такое рекурсия?
Способы организации рекурсивных функций}

\textbf{Рекурсия} -- это ссылка на определяемый объект во
время его определения.
Т. к. в Lisp используются рекурсивно определенные структуры (списки),
то рекурсия -- это естественный принцип обработки таких структур.

\paragraph{Способы организации рекурсивных функций}

\begin{itemize}
    \item Хвостовая рекурсия. В целях повышения эффективности рекурсивных
        функций рекомендуется формировать результат не на выходе из рекурсии,
        а на входе в рекурсию, все действия выполняя до ухода на следующий шаг
        рекурсии. Это и есть хвостовая рекурсия.
    \item Возможна рекурсия по нескольким параметрам
    \item Дополняемая рекурсия -- при обращении к рекурсивной функции
        используется дополнительная функция не в аргументе вызова , а вне его
    \item Выделяют группу функций множественной рекурсии. На одной
        ветке происходит сразу несколько рекурсивных вызовов. Количество
        условий выхода также может зависеть от задачи.
\end{itemize}

\section{Способы повышения эффективности реализации рекурсии}

\begin{itemize}
    \item Использование хвостовой рекурсии.
        Если условий выхода несколько, то надо думать о порядке их следования.
    \item Превращение не хвостовой рекурсии в хвостовую.
        Для превращения не хвостовой рекурсии в хвостовую и в целях
        формирования результата (результирующего списка) на входе в
        рекурсию, рекомендуется использовать дополнительные (рабочие)
        параметры. При этом становится необходимым создат фунецию --
        оболочку для реализации очевидного обращения к функции.
\end{itemize}
